%% Case 2: LaTeX Editor
\large \textbf{Case 2: Fine-grained Visual Counting and Semantic Filtering}

\begin{figure}[h!]
\centering
\includegraphics[width=0.65\textwidth]{cases/case2_code.png}
\caption{Screenshot of the Compiler}
\label{fig:case2_code}
\end{figure}

\begin{caseprompt}
The screenshot shows a snapshot of a LaTeX text editor a student uses for writing their PhD dissertation. What is the minimum number of chapter .tex files the student has opened?
\end{caseprompt}

\begin{casethinking}
So, let's look at the tabs at the top of the editor. \\
\textbf{$\dots$}\\
The visible tabs are:
\begin{itemize}[leftmargin=*]
    \item chapte$\dots$ (leftmost, maybe chapter1.tex as the active file in the center)
    \item chapte$\dots$ (next)
    \item thesis$\dots$ (this is thesis.tex, not a chapter)
    \item chapte$\dots$ (next)
    \item chapte$\dots$ (next)
    \item chapte$\dots$ (last one before the dark theme part)
\end{itemize}
\textbf{$\dots$}\\
To determine the minimum number of chapter .tex files opened, we analyze the tabs at the top of the editor.\\
1. Identify tabs: The image shows multiple tabs labeled chapte$\dots$ (truncated for "chapter$\dots$") and one tab labeled thesis$\dots$.  \\
2. Exclude non-chapter tabs: The thesis$\dots$ tab is not a chapter file, so we count only the chapte$\dots$ tabs. \\
3. Count chapter tabs: There are 5 visible chapte$\dots$ tabs (excluding the thesis$\dots$ tab).
\end{casethinking}

\begin{caseanswer}
5
\end{caseanswer}
